\section{Future Work}

There is a considerable amount of work that can be done to improve the ASV, as this was a prototype. During our design phase, we were limited by time and budget constraints, but future teams can build on our work to create a more robust and effective systems, especially if they can learn from our mistakes. The following sections will discuss some of the most important areas for improvement, including physical design, electronics, software, and testing. 

\subsection{Physical Design}

Several improvements to our prototype's mechanical structure should be considered for future ASV iterations:
\begin{itemize}
  \item \textbf{Optimized Part Geometry for Weight \& Strength:}  
    Our 3D-printed components were designed under tight time constraints and saw only a few iterations. If budget allows, future teams should perform wear tests on the 3D printed parts to see where they fail, and then redesign to optimize for weight and strength. There were a few prints where it was clear that the part was acceptable, but could use an easy redesign to improve the functionality. For example, the flange was one piece, but if it was split into two pieces, it would be easier to print and assemble.
  \item \textbf{Bracket Offset \& Mounting Security:}  
    The solar panel (27.5" wide) extended beyond our only one of four bolts could engage, leading to instability. This could be solved by redesigning the bracket geometry or increasing the offset distance to ensure all fasteners seat properly. Additionally, the panel support rods currently rely on friction and panel weight to stay in their mounts. Introduce captive set-screws or spring-loaded clips in each bracket to mechanically lock the rods in place, preventing slippage under harsh conditions    
  \item \textbf{Platform Material Stiffness:}  
    Our polycarbonate sheet platform flexed under load. A stiffer base, such as treated wood, aluminum, or even a thicker sheet of polycarbonate, would reduce deflection, improve stability, and better support mounted components.
  \item \textbf{Waterproofing at Fastener Joints:}  
    To preserve disassembly ease, we avoided epoxy and ran bolts through PVC fixtures, which allowed water ingress. Future designs should apply marine epoxy or silicone sealant at critical joints, or employ O-ring-sealed fastener assemblies, to balance rigidity, waterproofing, and maintainability. The mounting holes may have to be redesigned to accommodate this, as the current design doesn't allow for easy disassembly after the epoxy is applied.
\end{itemize}

\subsection{Electronics}

Several lessons from our prototype's wiring and power distribution should guide future iterations:
\begin{itemize}
  \item \textbf{Modular Motor Connections:} 
    Our motors were hard-wired via fully soldered leads, making removal and replacement cumbersome. Future designs should employ waterproof, keyed connectors inside the electrical box to allow rapid motor swap-outs for maintenance or upgrades without desoldering.
  \item \textbf{Reverse-Polarity Protection:} 
    During our demo, the battery pack was accidentally connected backwards, risking damage to electronics and breaking our ESC. Future designs should incorporate reverse-polarity protection—using an ideal-diode IC or MOSFET-based blocker in the main power feed—to automatically guard against destructive reverse connections and prevent user error.
  \item \textbf{Overcurrent \& Short-Circuit Safeguards:} 
    We lacked fusing or electronic breakers on our high-current rails. Future designs should include appropriately rated fuses or resettable circuit breakers on each major branch (motor, batteries, electronics) to isolate faults and protect wiring and components.
  \item \textbf{Power Distribution Board (PDB):} 
    Our prototype relied on wiring into a terminal junction bock, which increased bulk cabling and complicated troubleshooting. Future designs should use a custom PDB with integrated connectors, breakers, and status LEDs, with silk-screened labels marking each output (e.g.\ “Motor,” “MPPT,” “Servo”). A dedicated PCB, designed from the ground up, could consolidate off-the-shelf power components (amplifiers, signal generators, buck converters) into a single compact board, significantly reducing wiring complexity and system footprint.
\end{itemize}
    
\subsection{Software}

\subsection{Testing}