\section{Hardware}

The ASV's frame is built from two 4-inch diameter PVC pipes, each 28 inches long, mounted in parallel to provide flotation and structural support. A block of closed-cell foam is secured beneath the pipes to improve buoyancy and overall water stability. A 24-inch wide platform is built on top, consisting of two 1/8" MDF wooden sheets for rigidity and a polycarbonate sheet for mounting structural components. 

A waterproof enclosure is mounted at the center of the platform to protect onboard systems. Above it, a 100W solar panel is held in place by four custom 3D-printed vertical brackets. These mounts raise the panel above the deck to minimize shading and securely fasten it to the frame while allowing for easy removal or adjustment.

The motor is mounted at the stern using a custom 3D-printed flange. The flange is designed to fit a 1" PVC pipe which connects the motor to the frame, and keeps the PVC shaft vertical during operation. There are custom 3D printed brackets on either side of the PVC pipe to attach the motor to the PVC pipe on the bottom, and to attach the servo to the PVC pipe on the top. To keep the motor in place, two 3D printed bushing were designed to fit the PVC shaft, which keep the motor at a constant depth. The bushing rotate above the flange, but the coefficient of friction between the two is low enough to allow the motor to rotate freely. This design choice ensures that the motor remains securely in place during operation, even in rough water conditions.

Material choice for 3D printed parts was given careful consideration to ensure the ASV is lightweight, low cost, but also resistant to wear. The 3D printed Flange is made from ASA, a material known for its excellent UV resistance and mechanical properties. All other parts were made from PETG, which is a strong and durable material that has proven marine applications, making it suitable for the ASV's environment. The use of these materials ensures that the ASV can withstand the rigors of outdoor use while remaining lightweight and cost-effective.

All 3D-printed parts were designed to be lightweight, modular, and compatible with the layered structure of the boat, making them easy to replace or modify in future iterations.

