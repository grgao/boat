\section{Hardware}

The ASV's frame is built from two 4-inch diameter PVC pipes, each 28 inches long, mounted in parallel to provide flotation and structural support. A block of closed-cell foam is secured beneath the pipes to improve buoyancy and overall water stability. A 12-inch wide platform is built on top, consisting of a wooden plank for rigidity and a polycarbonate sheet for mounting structural components.

A waterproof enclosure is mounted at the center of the platform to protect onboard systems. Above it, a 100W solar panel is held in place by two custom 3D-printed vertical brackets. These mounts raise the panel above the deck to minimize shading and securely fasten it to the frame while allowing for easy removal or adjustment.

The motor is mounted at the stern using a custom 3D-printed flange that fits into a matching printed base. A rotating clamp mechanism wraps around the cylindrical motor body to hold it securely while allowing angle adjustment. A similar 3D-printed bracket is used to mount the rudder servo, ensuring proper alignment and stability during steering.

All 3D-printed parts were designed to be lightweight, modular, and compatible with the layered structure of the boat, making them easy to replace or modify in future iterations.

