\section{Operating Instructions}

Follow these steps to safely operate the Autonomous Surface Vehicle (ASV):

\begin{enumerate}
    \item \textbf{Power On:}
    \begin{itemize}
        \item Ensure the batteries are charged. The solar panel provides slow passive charging and should not be relied on for full charge.
        \item Inside the electronics enclosure, connect the battery wires. Each lead is clearly labeled for polarity—connect positive to positive and negative to negative.
        \item Verify that LEDs on the flight controller (SpeedyBee F405 V4) illuminate, indicating power to the system.
    \end{itemize}

    \item \textbf{System Initialization:}
    \begin{itemize}
        \item Wait a few seconds for the flight controller to initialize.
        \item Confirm GPS lock. The GPS module's LED (typically solid or blinking depending on model) indicates satellite connection.
    \end{itemize}

    \item \textbf{Manual Control (Optional):}
    \begin{itemize}
        \item Turn on the FlySky transmitter. Ensure the receiver on the ASV is powered and paired.
        \item When connected, you can use the transmitter to manually control throttle and rudder. This is useful for testing or override.
        \item Manual control is not required for autonomous operation but is available at any time.
    \end{itemize}

    \item \textbf{Autonomous Mission Launch:}
    \begin{itemize}
        \item Use iNav Configurator to upload or verify the waypoint mission.
        \item Use iNav Configurator to verify basic telemetry such as GPS status, battery voltage, and input channel readings.
        \item Confirm that all connections are secure and components are functioning.
        \item Arm the vehicle using the transmitter or iNav interface. The motor should begin spinning, and the boat can now be manually operated or left to drift as designed.
    \end{itemize}

    \item \textbf{Magnetic Field Activation:}
    \begin{itemize}
        \item The signal generator automatically powers on with the system and begins outputting a 40 Hz sine wave.
        \item The output is amplified and fed to the coil. If needed, use a clamp meter to verify that current is flowing.
    \end{itemize}

    \item \textbf{Monitoring:}
    \begin{itemize}
        \item Observe the ASV visually or via iNav telemetry if connected.
        \item Monitor GPS LED status and waypoint progress to confirm correct operation.
    \end{itemize}

    \item \textbf{Return to Home and Shutdown:}
    \begin{itemize}
        \item Use the transmitter or iNav to switch to “Return to Home” mode. The ASV will autonomously return to its starting location.
        \item Once the vehicle has returned and stopped moving, disarm it using iNav or the transmitter.
        \item Disconnect the battery by separating the labeled leads.
        \item Verify that all indicator LEDs are off and reseal the electronics enclosure to maintain waterproofing.
    \end{itemize}
\end{enumerate}

Before each deployment, double-check polarity labels, confirm dry connections, and verify GPS and flight controller status.
