\section{Introduction}

This project was developed to support the University of Wisconsin Missing in Action Recovery and Identification Project (UW MIA RIP), a program focused on locating and recovering U.S. service members who went missing during past wars. Many of these service members were lost when aircraft crashed in shallow coastal waters, especially in regions like Southeast Asia. Recovering their remains is difficult because traditional search methods are expensive and not always effective in those environments. Our team set out to design a low-cost, self-powered system that could help identify locations of underwater wreckage and assist with future recovery efforts.

The goal was to build an autonomous surface vehicle (ASV) that could generate a magnetic signal detectable by underwater equipment. The ASV works in tandem with an underwater autonomous vehicle (AUV), which has a 3-axis magnetometer on board to detect the magnetic field generated by the ASV. This magnetic signal provides a spatial reference point underwater, allowing the AUV to perform fine-grained magnetic mapping and search patterns beneath the ASV. By maintaining communication over a shared mission plan or synchronized search grid, the ASV and AUV work cooperatively to identify magnetic anomalies that may indicate buried aircraft debris or other metallic wreckage. 

While our ASV does not directly transmit control commands to the AUV, both platforms are designed to follow coordinated search paths based on GPS and mission parameters pre-planned by operators. This coordinated approach reduces surface search time and improves underwater mapping efficiency.

We were challenged to build an ASV that could be built on a budget of \$500, uses accessible materials, fits in a suitcase, and could operate independently in the field.

To meet these goals, we constructed the boat using affordable components like PVC pipes for flotation, foam for stability, and a combination of wooden and polycarbonate boards for mounting electronics. A solar panel powers the system by charging two batteries, which then provide energy to the onboard electronics. These electronics include a flight controller for navigation, a Raspberry Pi for control, and a signal generator and amplifier to drive a coil that emits the magnetic field.

The boat is designed to follow GPS waypoints on its own, while also allowing manual control if needed. Electronics are enclosed in a waterproof box to keep them safe during outdoor testing. We used prebuilt circuit boards where possible to reduce development time and focus on integrating everything into a working prototype.

This report describes the design process, how the system works, and what we learned throughout the semester. We also discuss the challenges we faced and how future teams can continue building on this foundation to improve the technology and support the mission of recovering those who served.
